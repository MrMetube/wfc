\section{Notizen}

\@todo(viktor):{}
\begin{enumerate}
    \item Passive Form: In dieser Arbeit, im Umfang dieser Arbeit, diese Arbeit...
    \item die Ergebnisse festhalten
    \item mehr Grafiken
    \item Tippfehler und Korrekturlesen
    \item Planare Graphen nur höchstens bei Voronoigraphen nennen und dann nie wieder
    \item Warum brauchen wir Heat. Welchen Effekt hat eine Heat von 1 auf z.B. Hexagongitter oder ''Rauschgraphen''
    \item \at{@translation} Collapsed und Observed als Begriffe vereinheitlichen
    \item zustände möglich superposition und zustandmenge alles vereinheitlichen und am anfang definieren
\end{enumerate}

\subsection{\at{@visual}}
\begin{enumerate}
    \item regelmäßig vs unregelmäßiges Gitter
    \item Delaunay Triangulierung und Voronoi Diagramme, dualer Graph
\end{enumerate}

\subsection{Must do}
\begin{enumerate}
    \item quadrat, regelmäßige und unregelmäßige gitter
    \item 8 statt 4 Directions in Extrahierung
    \item overlap / direction mask
    \item Heat und Heuristik zur verbesserung, erweiterungsmöglichkeiten mit simulated annealing und heating und cooling chance with heat
\end{enumerate}

\subsection{Can do}
\begin{enumerate}
    \item model synthesis
    \item Ähnlichkeit zum Input
    \item wahrscheinliche Lösbarkeit und NP-Hardness
    \item Nachbarschaft, Von Neumann-, Moore-
    \item Pitfalls in der Generierung von Graphen, einfach 8 nächste Nachbarn kann zu unsymmetrischen Benachbarungen führen
    \item solvability of a graph and NP-Hardness
\end{enumerate}