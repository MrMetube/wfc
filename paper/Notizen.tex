\section{Notizen}

\@todo(viktor):{}
\begin{enumerate}
    \item mehr Grafiken
    \item Planare Graphen nur höchstens bei Voronoigraphen nennen und dann nie wieder
    \item Warum brauchen wir Heat. Welchen Effekt hat eine Heat von 1 auf z.B. Hexagongitter oder ''Rauschgraphen''
    \item \at{@translation} Collapsed und Observed als Begriffe vereinheitlichen
    \item \at{@naming} Beispiel statt Eingabe, siehe Merrel mit Example
    \item \at{@naming} Überlappen und passen statt support
    \item \at{@naming} Heat statt strictness
    \item \at{@naming} für Output-Graph
    \item \at{@translation} für tiling und wrapping
    \item \at{@definition} für Beispiel und Pixel
    \item \at{@naming} zustände, mögliche zuständen, superposition und zustandmenge alles vereinheitlichen und am anfang definieren
\end{enumerate}

\at{@translation} ''Tile'' zu ''Bauteil''

\subsection{\at{@visual}}
\begin{enumerate}
    \item regelmäßig vs unregelmäßiges Gitter
    \item Delaunay Triangulierung und Voronoi Diagramme, dualer Graph
\end{enumerate}

\subsection{Must do}
\begin{enumerate}
    \item quadrat, regelmäßige und unregelmäßige gitter
    \item Heat und 
\end{enumerate}

\subsection{Can do}
\begin{enumerate}
    \item heat Heuristik zur lokalen anpassung, erweiterungsmöglichkeiten mit simulated annealing und heating und cooling chance with heat
    \item Ähnlichkeit zum Input, siehe Merrel, siehe section Regelextrahierung
    \item Nachbarschaft, Von Neumann-, Moore-
    \item Pitfalls in der Generierung von Graphen, einfach 8 nächste Nachbarn kann zu unsymmetrischen Benachbarungen führen
\end{enumerate}