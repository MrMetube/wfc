





\chapter{Zusammenfassung und Ausblick}
    Dieses Kapitel bietet einen Rückblick auf die erfolgreiche Erweiterung des Wave Function Collapse Algorithmus. Auch werden noch bestehende Schwachstellen der präsentierten Methodik erklärt. Diese könnten als Ansatz für weiterführende Forschung und Weiterentwicklung des Algorithmus dienen.
    
    \section{Fazit}
        Diese Arbeit zeigt, wie der Wave Function Collapse Algorithmus erweitert werden kann, damit die generierten Ausgaben nicht nur auf Gittern sondern auch auf, vom Nutzer gegebenen, Graphen geschehen kann. Dem Algorithmus muss hierfür lediglich ein Graph gegeben. Die Erfolgschance und Qualität der Ausgabe hängt dabei von den Eingenschaften des verwendeten Beispiels und des Graphen ab. Kommt es dazu, dass der Algorithmus nur weniger interessant Ausgaben generieren kann, so kann der Nutzer mittels des neu eingeführtem Konzept, der Heat, dem entgegenwirken, indem die erzielte Ähnlichkeit der Ausgabe zum Beispiel abgeschwächt wird.
    
        Am besten funktioniert der Algorithmus auf Graphen mit relativ uniformen Verteilungen von Zellen. Hierbei muss es sich nicht nur um ein einfaches Gitter handeln. Der Graph kann auch eine Zusammensetzung aus unterschiedlichen Arten von regelmäßigen Graphen sein, welche mittels des dargestellten Algorithmus zur Generierung von Graphen verbunden werden können. Je nach Beispiel können auch komplexere Graphen gute Ergebnisse liefern, doch kann es hier auch oft zu Fehlschlägen kommen, da nur eine Zelle in einem Graphen genügt um einen Widerspruch auszulösen. Um die Erfolgschancen einer Generierung zu verbessern, wurde präsentiert, wie der Algorithmus angepasst werden kann, um Backtracking zu erlauben. Dadurch kann bei einem Widerspruch ein Großteil der Arbeit erhalten bleiben, indem zu einem validen Punkt zurückgegangen wird, um von dort aus weiterzuarbeiten.
    
    
    
    \section{Beschränkungen}
        In diesem Abschnitt wird kurz auf die wichtigsten Beschränkungen des erweiterten Algorithmus und der Herangehensweise und Verwendung dessen eingegangen.
        
        Diese Arbeit hat ausschließlich mit zweidimensionalen Beispielen gearbeitet, obwohl Model Synthesis und Wave Function Collapse auch in 3D angewendet wurden. 
        
        Die Graphen wurden stets vor Beginn der Generierung erstellt. Der Algorithmus zur Generierung der Graphen erstellt dabei immer eine Triangulierung, also sind alle Zellen stark vernetzt mit naheliegenden Zellen. Dies bietete sich an, da Wave Function Collapse lokale Regeln aus den Beispielen auf die Ausgabe anwendet. Andererseits wäre es möglich auch anderen Arten von Graphen zu verwenden. So könnten relativ zur Triangulierung Kanten entfernt werden um Diskontinuitäten in der Ausgabe zu erzeugen. Andersherum könnten neue Kanten eingefügt werden, um entfernte Zellen miteinander zu verbinden, was in der Ausgabe zu brückenähnlichen Effekten führen könnte. 
        
        Desweiteren wurden uniform verteilte Zellen präferiert, da diese Erfahrungsgemäß eine bessere Erfolgsquote hatten. Dies hängt direkt damit zusammen, dass Heat als globaler Wert für alle Zellen genutzt wurde. Unregelmäßigere Anordnungen enthalten eher Zellen mit extremer Anzahl an Nachbarn oder anderen außergewöhnlichen Eigenschaften, wobei Zellen mit vielen Nachbarn mehr eingeschränkt sind und dadurch häufiger Widersprüche auslösen können. Haben alle Zellen relativ ähnliche Eigenschaften, so lässt sich leichter ein guter Heat-Wert für sie finden.
        
        Letztlich wurde die Heat immer manuell vom Nutzer vor Beginn festgelegt. Wurde sie während der Generierung angepasst, so verwarf der Algorithmus alle Arbeit und startete neu. Es wäre aber möglich, die Heat während des Ablaufs manuell oder automatisch z.B. mittels Heuristiken anzupassen und weiterzuarbeiten.
        


% ////////////////////////////////////////////////
\bibliographystyle{plain}
\bibliography{Literatur}



% ////////////////////////////////////////////////
\chapter*{Verwendetet Hilfsmittel}
\begin{table}[h]
    \centering
    \renewcommand{\arraystretch}{1.25}
    \begin{tabular}{|p{.33\linewidth}|p{.64\linewidth}|} 
        \hline \textbf{Hilfsmittel/Programm} & \textbf{Verwendung innerhalb der Arbeit}       \\
        \hline Visual Studio Code           & Code- und Texteditor                            \\
        \hline Odin-Programmiersprache      & Entwicklung der Anwendung                       \\
        \hline dearImgui                    & UI-Bibliothek für die Anwendung                 \\
        \hline raddebugger                  & Debugging der Anwendung                         \\
        \hline Git                          & Versionsverwaltung                              \\
        \hline Github                       & Backup und Synchronisierung der Arbeit          \\
        \hline Google Scholar               & Recherche wissenschaftlicher Literatur          \\
        \hline duck.ai, ChatGPT             & Brainstorming und Nachschlagen von Latex Syntax \\
        \hline \end{tabular}
    \end{table}
    


% ////////////////////////////////////////////////
\chapter*{Selbstständigkeitserklärung}

Durch meine Unterschrift erkläre ich, dass ich die vorliegende Arbeit mit dem Titel: \textit{Wave Function Collapse auf Graphen} selbständig verfasst und in gleicher oder ähnlicher Fassung noch nicht in einem anderen Studiengang als Prüfungsleistung vorgelegt habe. Ich habe alle von mir genutzten Hilfsmittel und Quellen, einschließlich generativer Modelle/KI angegeben und die den verwendeten Quellen und Hilfsmitteln wörtlich oder sinngemäß entnommenen Stellen in Form von Zitaten kenntlich gemacht. Darüber hinaus habe ich keine Hilfsmittel verwendet.

\vspace{3cm}

\noindent
\begin{minipage}[c]{5cm}
    \centering \hrulefill \\
    Ort, Datum
\end{minipage}
\hfill
\begin{minipage}[c]{7cm}
    \centering \hrulefill \\
    Unterschrift
\end{minipage}
