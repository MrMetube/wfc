\section{Umsetzung}

\subsection{Überlappung und Heat}
    \begin{itemize}
    \item Da Nachbarn nun in beliebigen Richtungen zu finden sind, gibt es kein direktes mapping zu den Supportregeln, welche ja je eine Menge pro Richtung sind.
    \item Dabei können wir entscheiden ob jeweils nur eine Menge oder mehrere Mengen betrachtet werden
    \item Ob eine Menge erlaubt ist, messen wir indem wir berechnen wir nahe die tatsächliche Richtung zu den Mengenrichtungen ist. Nun wählen wir die x-nähesten Mengen aus. Bei einer Strenge von 1 also die beste, bei 2 die zwei besten und so weiter.
    \item Die Strenge kann maximal 8 betrangen, da wir nur 8 "Himmelsrichtungen`` in der Extraktion genutzt haben. 
    \item Insgesamt bedeutet dies, dass ein Nachbar als z.B. im Norden oder im Nordosten betrachtet werden kann. Dadurch ist die menge an kompatiblen zuständen weniger eingeschränkt und es ist weniger wahrscheinlich, dass alle Zustände einer Zelle unmöglich werden
    \item Es folgt aber auch, dass nun Nachbar mit fast oder gar perfekter \\ übereinstimmung mit einer der Himmelsrichtung als nicht nur diese gelten, wodurch es zu einer Verzerrung im Output kommt.
    \item Wenn zuvor ein Muster im Input z.B. stehts N-S ausgerichtet war kann es ab einer Strenge vom 3 auch als (NW, N, NO)-(SW, S, SO) betrachtet werden. es könnte also auch als NW-SW betrachtet werden und aus einem Geradlinigen muster wird ein geknicktes oder zackiges.
    \item Aber genau diese auflockerung ist nötig, damit der Algorithmus mit \\ höherer Wahrscheinlichkeit eine Lösung findet
    \item 
    \item Wenn mehr Richtungen erlaubt sind, so kann dies auch betrachtet werden als dass nun nicht nur das Eingabe Muster sondern auch gedrehte Versionen der Eingabe möglich sind. 
    \item Nun können gedrehte regelmäßige Gitter auch gelöst werden. (Siehe Grid = Square with an angle ~27° and strictness 1 or 2).
    \item Auf unregelmäßigen Gittern kann man keine Drehung finden in der es einem regelmäßigen Gitter global gleich. Dennoch können Lösungen gefunden werden, da für jedes Zellenpaar/Benachbarung frei gewählt werden kann, ob eine solche gedacht Drehung angewendet wird oder nicht. 
    \item Es ist so, als würde der Algorithmus das Gitter lokal immer genau so verdrehen (mit Beschränkung durch die maximal Strictness), dass das tatsächliche Gitter lokal einem regelmäßigen gleicht.
    \item 
    \item Bei höherer Strictness und einer gelösten Zelle: ein Nachbar der Zelle kann nun aus mehr als einer Regelmenge entsprechend seiner Richtung wählen. Die Superposition des Nachbarn ist 'größer' bei höherer Strictness. Dadurch ist auch die Menge an Zuständen die Nachbarn des Nachbarn haben entsprechend größer, da weniger Beschränkungen auf sie einwirken. Es bleiben für einen längeren Zeitraum mehr Zustände möglich, wodurch ein Widerspruch durch eine Zelle mit leerer Superposition weniger wahrscheinlich ist. Im Gegenzug ist die lokale Ähnlichkeit zum Original verringert, da nun auch Nachbarn mit 'eindeutiger' Richtung zur Zelle auch einer weniger passenden Richtung zugewiesen werden können.
    \item Eine globale Strictness für alle Zellen hat einen negativen Effekt auf lokal regelmäßige Regionen eines Gitters und ein postiven Effekt auf sehr unregelmäßige Regionen des Gitters.
    \item Anstatt dass die Strictness global für alle Zellen zu Beginn bestimmt wird, kann man auch während des Collapse 'lernen' welche Regionen 'schwerer' zu lösen sind und dort die Strictness schrittweise erhöhen bis eine Lösung gefunden werden kann.
    \end{itemize}


\section{Ergebnisse und Diskussion}

\section{Fazit}
\subsection{Vergleich zum Original}
\subsection{Mögliche Anwendungsbereiche?}
\subsection{Ausblick und zukünftige Forschung}