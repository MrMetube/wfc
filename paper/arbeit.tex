\documentclass{article}
\usepackage[ngerman]{babel}
\usepackage{amsmath}

\begin{document}

\section{Dinge die besprochen werden müssen}
\begin{enumerate}
    \item prozedurale generierung
    \item wfc algorithmus erklären
    \item quadrat, regelmäßige und unregelmäßige gitter
    \item 8 statt 4 Directions in Extrahierung
    \item overlap lookup table
    \item overlap / direction mask
    \item Tilesets und Pixelmuster
    \item simulated annealing und heating und cooling chance with strictness
\end{enumerate}

\section{Dinge die besprochen werden können}
\begin{enumerate}
    \item model synthesis
    \item 
    \item Gittergenerierung
    \item Delaunay-Triangulierung
    \item Bowyer-Watson Algorithmus
    \item Voronoi Diagramme
    \item Townscapers Gitter
    \item 
    \item Backtracking
    \item Constraint Solving
    \item 
    \item Vergleiche zwischen Algorithmen und Daten
    \item Ähnlichkeit zum Input
    \item wahrscheinliche Lösbarkeit
    \item Laufzeit und Speicher, Effizienz
    \item State Count vs. Entropy
    \item Random jitter auf Entropy?
    \item 
    \item die App und ihre Features
\end{enumerate}

\pagebreak
\tableofcontents
\pagebreak



\section{Einleitung}
% \section{Motivation}
% \section{Problemstellung und Ziel}
% \section{Problemlösung}
% \section{Aufbau der Arbeit}

\section{Model Synthesis - Paul Merrell}
\begin{itemize}
    \item Original Idea based on texture synthesis
    \item is this true? discrete version of continous texture synthesis
    \item Rough overview of the methodology
    \item Key advantages over previous methods
\end{itemize}

\section{Wave Function Collapse - Maxim Gumin}
\begin{itemize}
    \item WFC - Renaming and popularization
    \item Overlapping tiles in extraction and for overlap checking
    \item search for lowest entropy (find explaination for added noise in code)
\end{itemize}

\section{Oskar Stålberg}
\begin{itemize}
    \item successful application for 3D use case
    \item aperiodic infinite deterministic irregular relaxed quadrilateral grids
    \item but still mostly regular grids of quads and not fully irregular
\end{itemize}



\section{Begriffe}

@todo(viktor): Hier Visualisierungen für
\begin{itemize}
\item regelmäßig vs unregelmäßiges Gitter
\item Delaunay Triangulierung
\item Voronoi Diagramme
\end{itemize}

\section{Wave Function Collapse}
\begin{itemize}
\item wfc arbeitet ursprünglich mit quadratgittern hierbei sind die position jeder zelle und dessen nachbar zellen durch das gitter definiert und können zB. mittels 2D Indizes identifiziert werden. Die Nachbar sind dann über je einen schritt in eine der 4 Richtungen im Gitter, siehe 2 Achsen je positiv oder negativ, erreicht werden
\item allgemeiner kann jeder zelle eine 2D Koordinate zugewiesen werden und nun sind die nachbar stets durch je einen der zwei Basisvektoren des Koordinatensystems zu erreichen, auch hier positiv oder negativ
\item Nun auf unregelmäßigen gittern sind zellen frei in der Ebene des 2D Koordinatensystems angeordnet. Nun muss die Menge der Nachbarn für jede Zelle berechnet werden
\item Wie im regelmäßigen gitter wollen wir, dass wenn zwei Zellen benachbart sind, dass sie eine Kante Teilen, also tatsächlich eine Verbindung besteht. für unregelmäßige gitter können wir ein Voronoi Diagramm erstellen, wodurch die Ebene in Regionen aufgeteilt wird die je alle Punkte umfässt die einer unserer Zelle am nächsten sind. 
\item Um dieses Voronoi diagramm zu erstellen gibt es mehrere Wege. Einer ist, eine Delaunay Triangulierung zu generieren, da dies die Inverse des Voronoi Diagramm darstellt. Die Zellen im Voronoi Diagramm entsprechen den Ecken in der Delaunay Triangulierung und die Kanten den Seitenhalbierenden der Kanten zwischen den Ecken. Aus der Delaunay Triangulierung lassen sich auch die Nachbarn einer Zelle direkt auslesen, da es einfach alle Punkte sind die mit dem Mittelpunkte der Zelle eine Kante teilen.
\end{itemize}

\section{Delaunay Triangulation - Bowyer-Watson Algorithmus}
\begin{itemize}
    \item Defintion der Triangulierung
    \item Algorithmus kurz erklären
    \item Umwandlung zu Voronoi Diagramm
    \item Voronoi wie - vertices -> cell centers, edges -> perpendicular bisector as vertex, edges between in (clockwise) order
\end{itemize}

\section{Voronoi Diagramm}
\begin{itemize}
    \item Defintion
    \item nützliche Eigenschaften
    \item unendliche Randzellen begrenzen
\end{itemize}

\section{Einschränkung des Algorithmus und Idee zur Erweiterung}

Die ursprüngliche Form des Wave Function Collapse nimmt in 2D Pixelmuster und Tilesets als Eingabe und produziert darauß wieder 2D Muster. 
Pixel liegen stets auf einem quadratischen Gitter. Bei Tilesets ist die grafische Gestaltung des Tiles uneingeschränkt. Dennoch sind die Tiles selbst quadratisch. Dies schränkt die Gestaltung des Inhalts der Tiles in sofern ein, dass die Kanten zu anderen Kanten passen müssen. Auch bei 3D Inputs werden die Modelle in blockförmige Bausteine zerschnitten, damit der Collapse auf einem 3D Gitter von Würfeln arbeiten kann.

Man beschränkt sich auf regelmäßige quadratische Gitter, da ihre Struktur bestimmte Vorteile mit sich bringt. Benachbarung von Zellen ist implizit durch die Struktur des Gitters mitgeliefert. Eine Zelle des Ausgabegitters ist durch seine Koordinaten im Gitter definiert und seine Nachbarn lassen sich einfach durch Schritte in vier festen Richtungen in 2D oder sechs Richtungen in 3D finden. Diese Eigenschaft gilt durch die Regelmäßigkeit des Gitters für alle Zellen gleichermaßen. Dies ist nützlich, da die Benachbarungsregeln, die aus der Eingabe herausgezogen werden, eben auch auf einem regelmäßigem Gitter basieren und direkt im Algorithmus verwendet werden können.

Der Nachteil dadurch, dass die Generierung immer auf einem regelmäßigem quadratischen Gitter läuft, ist, dass die Ausgabe im Ganzen Artefakte des Gitter aufweist. Vertikale und horizontale Linien können im Pixelgitter dargestellt werden, aber bei diagonalen oder frei geformte Kurven und Linien müsste eine Zelle des Ausgabegitters anteilmäßig ausgefüllt sein, was aber nicht möglich ist. Eine Zelle kann immer nur einen diskreten finalen Zustand wählen. Es kommt zu Aliasing.

Wenn man sich den Kern des Wave Function Collapse Algorithmus genauer anguckt, erkennt man, dass die Entscheidung, welchen Zustand eine Zelle erhällt, nur von den möglichen Zuständen ihrer Nachbarn abhängt. Deren Zuständ geschränken die Menge aller in der Eingabe existierenden Zustände, auf genau diese, die in der Eingabe nebeneinander gefunden wurden. Dabei ist auf dem Gitter immer eindeutig, dass z.B. die Zelle N nördlich einer betrachteten Zelle M, nur Zustände wählen kann die in der Eingabe nördlich von einem von M's Zuständen gefunden wurde. Damit der Algorithmus die Menge an möglichen Zuständen einer Zelle finden kann, braucht es also nur eine Funktion die aus der tatsächlichen Richtung von M zu N die betrachtete Richtung der Benachbarungregeln ergibt. Im Gitter ist gibt diese Funktion meißt die Identität; die Richtung zum Nachbarn ist die Richtung für die Regeln. 
% \newline
% $$\text{cell}=(x_c,y_c),\ \text{neighbour}=(x_n,y_n)$$
% $$\mathbf{d} \;=\; \text{neighbour} - \text{cell}$$
% Für das normale quadratische Gitter:
% $$\mathbf{d} \;=\; (x_n-x_c,\; y_n-y_c)$$

% $$Gitter(\text{cell},\text{neighbour}) = 
% \begin{cases} 
%     \text{Osten} & \text{falls } d_x = 1,\ d_y = 0 \\
%     \text{Norden} & \text{falls } d_x = 0,\ d_y = 1 \\
%     \text{Westen} & \text{falls } d_x = -1,\ d_y = 0 \\
%     \text{Süden} & \text{falls } d_x = 0,\ d_y = -1 \\
% \end{cases}$$


% Für das torusartige Gitter:
% $$\mathbf{W} \;=\; Gitterbreite,\ \mathbf{H} \;=\; Gitterh"ohe$$

% $$d = \operatorname{wrap}\big((x_n,y_n)-(x_c,y_c), (W,H)\big)$$


% $$\operatorname{wrap}(\Delta,M)=\bigl((\Delta \bmod M)+M\bigr)\bmod M$$

% $$Gitter_{torus}(\text{cell},\text{neighbour}) = 
% \begin{cases} 
%     \text{Osten} & \text{falls } d_x = 1,\ d_y = 0 \\
%     \text{Norden} & \text{falls } d_x = 0,\ d_y = 1 \\
%     \text{Westen} & \text{falls } d_x = W-1,\ d_y = 0 \\
%     \text{Süden} & \text{falls } d_x = 0,\ d_y = H-1 \\
% \end{cases}$$


Wenn man mit WFC eine Ausgabe erstellen will, die sich selbst tiled(also linker Rand gleich rechter Rand und so) kann man eine imaginäre Kante von einer Zelle am rechten Rand zu einer Zelle am linken Rand erstellen und dem Algorithmus definieren dass diese Benachbarung in Richtung Osten geht. Nun sind die Ränder des Gitters benachbart und das Ergebnis ist, dass die Ränder, wenn eine Lösung gefunden wird, zueinander passen. Wenn man dies für einen Rand macht, so ist die Form der Zellen mit ihren Nachbarn nicht mehr eine flache Ebene, sondern die Oberfläche eines Zylinders(ohne Deckel und Boden). Verbindet man auf die selbe Weise auch den anderen Rand, ist die Form nun ein Torus. Der Algorithmus selbst muss also nur auf diese Weise angepasst werden, um auf anderen Anordnungen von Zellen zu funktionieren.

@todo(viktor): Müssen es planare Graphen sein?

@incomplete Diagonale Regeln, besseres Sampling der Eingabe, Regelmengen, höhere Entropie durch Strictness

\section{Wave Function Collapse auf planaren Graphen}

@incomplete Welche Formel für Richtung? Mehr als eine Richtung. Welche Heuristik für Strictness? Warum auch Diagonalen extrahieren?
@incomplete Welche Nachbarn hat eine Zelle? Wie beeinflusst eine Zelle seine Nachbarn und was hat Heat damit zutun? Zustand/Richtungsmengen erklären. Overlap Lookup Table

\section{State Extraction}
\begin{itemize}
    \item Bei der Extrahierung nutze ich die Methode mit Überlappung, d.h. von jedem Pixel in dem Eingabebild wird jeweils ein N-mal-N großes Fenster betrachtet. 
    \item Je nach Eingabebild kann ein Wrapping an den Kanten gewünscht oder unerwünscht sein. Ist es erlaubt, dann geht bei Pixeln die Näher im Rand sind als N Pixel das betrachtete Fenster über den Bildrand hinaus und zurück zur gegenüberliegeden Kante und von da weiter. So als wäre das Eingabebild an dieser Kante nocheinmal angelegt. Ist es nicht erlaubt, so werden für solche Pixel keine Fenster betrachtet.
    \item Ich habe die gesamte Zeit stets N = 3 benutzt. Meine Auswahl an Eingabebildern wurden dafür konzipiert und geben damit die ''besten`` Ergebnisse.
    \item Wenn zwei Fenster exakt die gleichen Farbwerte haben, dann werten wir sie als ein und den selben und notieren dass der Zustand eine höhere Frequenz in der Eingabe hat. Diese Frequenz wird bei der späteren Generierung mit in betracht gezogen.
    \item Der mittlere Pixel stellt den Farbwert dieses Zustands dar. 
    \item Mit den Pixeln drum herum errechnen wir, welche dieser Fenster nun überlappen können.
    \item @todo(viktor): 8 richtungen
\end{itemize}

\section{Overlap Lookup Table}
\begin{itemize}
    \item Für den Wave Function Collapse muss ich später prüfen, ob ein Zustand neben einem (oder mindestens einen von vielen) Zuständen seien darf.
    \item Hierfür erstelle ich eine Tabelle mit drei Axen. Die ersten beiden sind die zu vergleichenden Zustände, die dritte bestimmt die Himmelsrichtung entlang der ich diese vergleiche.
    \item Jeder Eintrag sagt mir nun, ob ein Zustand A neben einem Zustand B der in Richtung D von A aus liegt in der Eingabe existiert hat und somit möglich ist.
    \item Da die Überlappung eine symmetrische Eigenschaft ist, ist auch die Tabelle symmetrisch. Man könnte also nur halb so viele Einträge speichern solange man dies beim auslesen bedenkt. In meiner Implementierung habe ich diese Optimierung nicht genutzt um das Auslesen möglichst einfach zu halten, da selbst meine komplexesten Eingaben nie mehr als 600 Zustände hatten und die Tabelle somit nie mehr als ca. 300 kB groß wahr. Der Median liegt bei 77 Zuständen also 5 kB.
    % 12 20 31 37 41 50 51 55 61 66 77 78 107 111 117 124 153 259 423 526
    \item @todo(viktor): Visuelle Darstellung
    \item Die Generierung dieser Tabelle läuft wie folgt:
    \item Ich vergleiche jeden Zustand mit sich selbst und jedem anderem Zustand indem ich für jede Himmelsrichtung prüfe ob die Farbwerte in dem Teil des Fensters zu einander identisch sind.
    \item Zum Beispiel nehmen wir den oberen-linke 2x2 Ausschnitt von Zustand A und vergleichen ihn mit dem unteren-rechten 2x2 Ausschnitt von Zustand B. Sind diese gleich so kann B im Nordwesten von A sein oder andersherum A im Südosten von B.
\end{itemize}

\section{Prozedurale Voronoi-Graphen}
@todo(viktor): Erklärung der Generierung separat zur erklärung der Anforderung an den Graphen auf dem der Algorithmus laufen soll  
\begin{itemize}
    \item Der Algorithmus arbeitet auf einem gegebenen Graphen. Die Verteilung der Zellen kann regelmäßig oder ohne Struktur sein. Es muss kann auch über den Graphen hinweg variieren. Jede Zelle hat eine 2D Position und jeder Zelle müssen alle ihre Nachbarzellen zugewiesen werden. Zellen sind benachbart, wenn sie eine Kante teilen. Eine Kante sind alle Punkte die gleichweit von zwei Zellenmittelpunkten entfernt sind. 
    \item In meiner Anwendung erstelle ich die Graphen prozedural aus einfachen Mustern oder mittels Zufallsprinzip. 
    \item Meine generierten Graphen sind immer planar, da ich immer dem selben Ablauf folge. Erst erstelle ich eine Menge an 2D Punkten, die wie gesagt eine beliebige Anordnung habe. Dann erstelle ich auf diesen Punkten eine Delaunay-Triangulierung. Das Gegenstück einer solchen Triangulierung ist ein Voronoi-Diagramm. Man findet es in dem man die Eckpunkte der Dreiecke als Mittelpunkte der Voronoizellen nimmt und die Kanten der Dreiecke enden in den Nachbarzellen jeder Zelle. Da Zellen am Rand von diesem Voronoi-Diagramm nach außen unendlich groß sind, begrenzen wir diese Zellen auf eine freigewählte rechteckige Fläche. 
    \item Zu Beginn des Wave Function Collapse werden alle Zellen mit einer Superposition aller möglichen Zustände initialisiert.
\end{itemize}

\section{Collapse Cells}
\begin{itemize}
    \item what are steps?
    \item the step control flow diagram?
    \subitem add "loop until the grid is fully collapsed"
    @todo(viktor): Erst den allgemeinen Ablauf ohne Backtracking und dann erklären wo und wie Backtracking eingebaut werden kann 
\end{itemize}

\subsection{Backtracking}
\begin{itemize}
    \item Goal, dont restart when reaching a contradiction, most of the time most of the cells are fine, only a small number of cells cause a contradiction
    \item Key Observation: for each cell the set of possible states only ever shrinks, as more and more constraints are placed on it. We never need to add back a state.
    \item For each state there is only ever one point in time when it is removed. all we need to know to advance is know which steps are removed right now.
    \item the naiive implementation creates a set of states and removes each state when it becomes impossible
    \item but We dont need to remove entries from this set, we can just store at which step in time each state became impossible and whenever we check all remaining states we ignore states that where "removed" before the current step happened
    \item Implementation: for each state in a cell store a sentinel value as the step index of removal. in the context store the current step index. after each step, increment the step index. then when propagating changes whenever a state is removed, store the current steps index as its removal timestamp. 
    \item Now we can rewind by only changing the context current step by some desired amount. the condition that checks if a state is removed now ensures that rewound removals can be detected and ignored.
\end{itemize}

\section{Datenstrukturen und Algorithmen}
\subsection{Overlap - Direction Mask / State Buckets?}
\begin{itemize}
    \item @todo(viktor): Strictness muss ich nicht mit der Größe der Zahl im Code hier gleichstellen, dass ist nur verwirrend wenn 'hohe' strictness eigentlich weniger strenge bedingungen darstellt
    \item Da Nachbarn nun in beliebigen Richtungen zu finden sind, gibt es kein direktes mapping zu den Supportregeln, welche ja je eine Menge pro Richtung sind.
    \item Dabei können wir entscheiden ob jeweils nur eine Menge oder mehrere Mengen betrachtet werden
    \item Ob eine Menge erlaubt ist, messen wir indem wir berechnen wir nahe die tatsächliche Richtung zu den Mengenrichtungen ist. Nun wählen wir die x-nähesten Mengen aus. Bei einer Strenge von 1 also die beste, bei 2 die zwei besten und so weiter.
    \item Die Strenge kann maximal 8 betrangen, da wir nur 8 "Himmelsrichtungen`` in der Extraktion genutzt haben. 
    \item Insgesamt bedeutet dies, dass ein Nachbar als z.B. im Norden oder im Nordosten betrachtet werden kann. Dadurch ist die menge an kompatiblen zuständen weniger eingeschränkt und es ist weniger wahrscheinlich, dass alle Zustände einer Zelle unmöglich werden
    \item Es folgt aber auch, dass nun Nachbar mit fast oder gar perfekter \\ übereinstimmung mit einer der Himmelsrichtung als nicht nur diese gelten, wodurch es zu einer Verzerrung im Output kommt.
    \item Wenn zuvor ein Muster im Input z.B. stehts N-S ausgerichtet war kann es ab einer Strenge vom 3 auch als (NW, N, NO)-(SW, S, SO) betrachtet werden. es könnte also auch als NW-SW betrachtet werden und aus einem Geradlinigen muster wird ein geknicktes oder zackiges.
    \item Aber genau diese auflockerung ist nötig, damit der Algorithmus mit \\ höherer Wahrscheinlichkeit eine Lösung findet
    \item 
    \item Wenn mehr Richtungen erlaubt sind, so kann dies auch betrachtet werden als dass nun nicht nur das Eingabe Muster sondern auch gedrehte Versionen der Eingabe möglich sind. 
    \item Nun können gedrehte regelmäßige Gitter auch gelöst werden. (Siehe Grid = Square with an angle ~27° and strictness 1 or 2).
    \item Auf unregelmäßigen Gittern kann man keine Drehung finden in der es einem regelmäßigen Gitter global gleich. Dennoch können Lösungen gefunden werden, da für jedes Zellenpaar/Benachbarung frei gewählt werden kann, ob eine solche gedacht Drehung angewendet wird oder nicht. 
    \item Es ist so, als würde der Algorithmus das Gitter lokal immer genau so verdrehen (mit Beschränkung durch die maximal Strictness), dass das tatsächliche Gitter lokal einem regelmäßigen gleicht.
    \item 
    \item Bei höherer Strictness und einer gelösten Zelle: ein Nachbar der Zelle kann nun aus mehr als einer Regelmenge entsprechend seiner Richtung wählen. Die Superposition des Nachbarn ist 'größer' bei höherer Strictness. Dadurch ist auch die Menge an Zuständen die Nachbarn des Nachbarn haben entsprechend größer, da weniger Beschränkungen auf sie einwirken. Es bleiben für einen längeren Zeitraum mehr Zustände möglich, wodurch ein Widerspruch durch eine Zelle mit leerer Superposition weniger wahrscheinlich ist. Im Gegenzug ist die lokale Ähnlichkeit zum Original verringert, da nun auch Nachbarn mit 'eindeutiger' Richtung zur Zelle auch einer weniger passenden Richtung zugewiesen werden können.
    \item Eine globale Strictness für alle Zellen hat einen negativen Effekt auf lokal regelmäßige Regionen eines Gitters und ein postiven Effekt auf sehr unregelmäßige Regionen des Gitters.
    \item Anstatt dass die Strictness global für alle Zellen zu Beginn bestimmt wird, kann man auch während des Collapse 'lernen' welche Regionen 'schwerer' zu lösen sind und dort die Strictness schrittweise erhöhen bis eine Lösung gefunden werden kann.
\end{itemize}

\subsection{vielleicht was zur App}

\section{Ergebnisse}
\begin{itemize}
    \item Bilder
    \item Base Case - reguläres Gitter
    \item geringe Striktheit - Richtung verliert einfluss auf output und es kommen nur noch adjazenz regeln?
    \item Lokale Ähnlichkeit im Vergleich zum Input
    \item Laufzeit und Speicheranalyse?
\end{itemize}

\section{Fazit}

\subsection{Vergleich zum Original}

\subsection{Mögliche Anwendungsbereiche?}

\subsection{Ausblick und zukünftige Forschung}

\begin{itemize}
    \item{Kann dies auf 3D angewendet werden?}
    \item{Welche Beziehung besteht zwischen Input Muster und Gitter und strictness? Linien und Flächen im Input}
    \item{Welche Eigenschaften eines Gitters bestimmen die Wahrscheinlichkeit eine Lösung zu finden?}
\end{itemize}

\section{Literaturverzeichnis}

\end{document}
