\begin{algorithm}
    \caption{Bowyer-Watson Algorithmus \cite{bowyer, watson}}
    \label{alg:bowyer_watson}
        
    \begin{enumerate}
        \item Gegeben sei eine Menge an Punkte
        \item Erstelle das Super-Dreieck so dass es jeden Punkt enthält
        
        \item Füge jeden Punkt schrittweise in die Triangulierung ein: \begin{enumerate}
            \item Finde die Dreiecke, in dessen Umkreis der Punkt liegt
            \subitem Diese Dreiecke können nicht zur Triangulierung gehören
            \item Sammel alle Kanten dieser Dreiecke
            \subitem Gibt es eine Kante nur einmal, ist sie Teil der Hülle der Dreiecke
            \subitem Eine Kante, die sich zwei Dreiecke teilen, liegt innerhalb der Hülle
            \item Erstelle eine neue Kante für jede Ecke der Hülle zum eingefügten Punkt
            \item Erstelle neue Dreiecke aus diesen Kanten
            \item Füge die neuen Dreiecke in die Triangulierung ein
        \end{enumerate}
        
        \item Entferne alle Dreiecke, die eine Kante mit dem Super-Dreieck teilen
    \end{enumerate}
        
\end{algorithm}