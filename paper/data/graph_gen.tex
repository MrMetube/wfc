\begin{figure}[H]
    \centering
    \begin{minipage}{\linewidth}
        \rule{\linewidth}{0.4pt}
        
        \begin{enumerate}
        \item Erstelle eine Menge an Punkten
        
        \item Generiere die Delaunay-Triangulierung \cite{bowyer} \cite{watson} \at{@visual} \begin{enumerate}
            \item Erstelle das Super-Dreieck so dass jeder Punkt innerhalb liegt
            
            \item Füge jeden Punkt schrittweise in die Triangulierung ein: \begin{enumerate}
                \item Finde die Dreiecke in dessen Umkreis der Punkt liegt
                \subitem Ist der Punkt innerhalb des Umkreises kann dieses Dreieck nicht zur finalen Triangulierung gehören
                \item Sammel alle Kanten dieser Dreiecke
                \subitem Kommt eine Kante nur einmal vor, ist sie Teil der Hülle dieser Dreiecke
                \subitem Alle anderen Kanten sind innerhalb dieser Hülle, da sich zwei Dreiecke diese Kante teilen
                \item Erstelle eine neue Kante zum eingefügten Punkt für jede Ecke der Hülle
                \item Erstelle neue Dreiecke aus diesen Kanten und der Hülle
                \item Füge die neuen Dreiecke in die Triangulierung ein
            \end{enumerate}
            
            \item Enferne alle Dreiecke die eine Kanten mit dem Super-Dreieck teilen
            \item Die Ecken und Kanten aller Dreiecke bilden den Graphen
            \subitem Jede Ecke wird der Mittelpunkt einer Voronoizelle
            \subitem Die Kanten zeigen welche Zellen benachbart sind
        \end{enumerate}
        
        \item Erstelle die Voronoizellen aus der Triangulierung \begin{enumerate}
            \item Finde alle Dreiecke einen Punkt teilen
            \item Die Schwerpunkte der Dreiecke sind die Eckpunkte der Zelle
            \item Verbinde die Eckpunkte der Zelle zu Kanten
            \item Begrenze die Voronoizellen auf den gewünschten Bereich \ref{fig:voronoi_clipping}
        \end{enumerate}
        
    \end{enumerate}
        
    \rule{\linewidth}{0.4pt}
    \end{minipage}
    
    \caption{Algorithmus zur Generierung der Graphen}
    
    \label{fig:graph_gen}
\end{figure}