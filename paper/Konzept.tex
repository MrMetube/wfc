\section{Konzept}

\subsection{Einschränkung des Algorithmus}
\begin{itemize}
\item bisher nur auf regelmäßigen gittern
\item was ist mit Townscaper? lies das Paper zum Gitter!
\end{itemize}

\subsection{Kernidee: WFC kann auf Graphen aller Art laufen}
\begin{itemize}
\item wfc prüft regeln nur lokal und propagiert die entscheidungen auch nur immer an die nächsten nachbar zellen
\item somit ist die topologie des zellennetz in der propagierung egal
\item nur beim prüfen der zellen muss auf die nun auf die frei variierende richtung zum nachbar geachtet werden
\item vorher war klar welcher nachbar in nswo liegt, da regelmäßiges gitter, daraus war klar welche regeln zum prüfen relevant waren
\item nun muss die menge an relevanten regeln je nach zellen paar / nachbarschaft ermessen werden
\end{itemize}

\subsection{Erweiterung des Algorithmus}
\begin{itemize}
\item extraktion nun mit diagonalen regeln
\item warum? besseres sampling des inputs also der 2d funktion die die eingabe darstellt

\item beim prüfen wird nun möglicherweise mehr als eine Regelmenge betrachtet
\end{itemize}