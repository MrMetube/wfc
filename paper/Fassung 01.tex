\section{Hintergrund}

\subsection{Stand der Technik}

\subsubsection{WFC Algorithm}

    % \url{https://github.com/mxgmn/WaveFunctionCollapse?tab=readme-ov-file#algorithm}

    \begin{enumerate}
    \item Read the input bitmap and count NxN patterns.
    \subitem (optional) Augment pattern data with rotations and reflections.
    \item Create an array with the dimensions of the output (called ''wave'' in the source). Each element of this array represents a state of an NxN region in the output. A state of an NxN region is a superposition of NxN patterns of the input with boolean coefficients (so a state of a pixel in the output is a superposition of input colors with real coefficients). False coefficient means that the corresponding pattern is forbidden, true coefficient means that the corresponding pattern is not yet forbidden.
    \item Initialize the wave in the completely unobserved state, i.e. with all the boolean coefficients being true.
    \item Repeat the following steps:
    \subitem Observation:
    \subsubitem Find a wave element with the minimal nonzero entropy. If there is no such elements (if all elements have zero or undefined entropy) then break the cycle (4) and go to step (5).
    \subsubitem Collapse this element into a definite state according to its coefficients and the distribution of NxN patterns in the input.
    \item Propagation: propagate information gained on the previous observation step.
    \item By now all the wave elements are either in a completely observed state (all the coefficients except one being zero) or in the contradictory state (all the coefficients being zero). In the first case return the output. In the second case finish the work without returning anything.
    \end{enumerate}


\subsection{Gitter, Graphen, Triangulierung, Voronoi}
    \url{https://en.wikipedia.org/wiki/Delaunay_triangulation}
    \url{https://en.wikipedia.org/wiki/Voronoi_diagram}
    \url{https://de.wikipedia.org/wiki/Polygonnetz}
    \url{https://en.wikipedia.org/wiki/Triangulated_irregular_network}
    \url{https://en.wikipedia.org/wiki/Lattice_graph}
    
    \subsubsection{Gitter}
    \subsubsection{Graphen}
    \subsubsection{Triangulierung, Quadrilierung}
    \subsubsection{Delaunay Triangulation}
        \begin{itemize}
        \item Defintion der Triangulierung
        \item Bowyer-Watson Algorithmus
        \item Umwandlung zu Voronoi Diagramm:
        \item vertices -> voronoi cell centers
        \item edges -> cell neighbours
        \item perpendicular bisector of edges -> cells vertices on its boundary
        \end{itemize}
    \subsubsection{Voronoi Diagramm}
        \begin{itemize}
        \item Definition
        \item nützliche Eigenschaften
        \end{itemize}

\section{Ergebnisse und Diskussion}
    
    \begin{itemize}
    \item Bilder
    \item Base Case - reguläres Gitter
    \item geringe Striktheit/höhere Heat - Richtung verliert einfluss auf output, lokale Verzerrungen dominieren, sobald entgegengesetze richtungen wählbar sind dominiert Adjazenz
    \item Lokale Ähnlichkeit im Vergleich zum Input
    \item Pixel/Linienmuster und Flächenmuster
    \item Zu kleiner Lösungsraum ergibt uniformen Output, nur eine Farbe die flächenartig ist.
    \end{itemize}
    
\section{Fazit}

\subsection{Vergleich zum Original}

\subsection{Mögliche Anwendungsbereiche?}

\subsection{Ausblick und zukünftige Forschung}

\begin{itemize}
    \item{Kann dies auf 3D angewendet werden?}
    \item{Welche Beziehung besteht zwischen Input Muster und Gitter und strictness? Linien und Flächen im Input}
    \item{Welche Eigenschaften eines Gitters bestimmen die Wahrscheinlichkeit eine Lösung zu finden?}
\end{itemize}