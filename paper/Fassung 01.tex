\section{Hintergrund}

\subsection{Gitter, Graphen, Triangulierung, Voronoi}
    \url{https://en.wikipedia.org/wiki/Delaunay_triangulation}\\
    \url{https://en.wikipedia.org/wiki/Voronoi_diagram}\\
    \url{https://de.wikipedia.org/wiki/Polygonnetz}\\
    \url{https://en.wikipedia.org/wiki/Triangulated_irregular_network}\\
    \url{https://en.wikipedia.org/wiki/Lattice_graph}\\
    
    \subsubsection{Gitter}
    \subsubsection{Graphen}
    \subsubsection{Triangulierung, Quadrilierung}
    \subsubsection{Delaunay Triangulation}
        \begin{itemize}
        \item Defintion der Triangulierung
        \item Bowyer-Watson Algorithmus
        \item Umwandlung zu Voronoi Diagramm:
        \subitem vertices $\rightarrow$ voronoi cell centers
        \subitem edges $\rightarrow$ cell neighbours
        \subitem perpendicular bisector of edges $\rightarrow$ cells vertices on its boundary
        \item \at{@visual}
        \end{itemize}
    \subsubsection{Voronoi Diagramm}
        \begin{itemize}
        \item Definition
        \item nützliche Eigenschaften
        \end{itemize}


        
\section{Ergebnisse und Diskussion}
    \begin{itemize}
    \item Bilder
    \item Base Case - reguläres Gitter
    \item höhere Heat - Richtung verliert einfluss auf output, lokale Verzerrungen dominieren, sobald entgegengesetze richtungen wählbar sind dominiert Adjazenz
    \item Lokale Ähnlichkeit im Vergleich zum Input
    \item Pixel/Linienmuster und Flächenmuster
    \item Zu kleiner Lösungsraum ergibt uniformen Output, nur eine Farbe die flächenartig ist.
    \end{itemize}

    
    
\section{Fazit}

\subsection{Vergleich zum Original}
\subsection{Mögliche Anwendungsbereiche?}
\subsection{Ausblick und zukünftige Forschung}
    \begin{itemize}
    \item Kann dies auf 3D angewendet werden?
    \item Welche Beziehung besteht zwischen Input Muster und Gitter und strictness? Linien und Flächen im Input
    \item Welche Eigenschaften eines Gitters bestimmen die Wahrscheinlichkeit eine Lösung zu finden?
    \item mögliche Heuristiken für Heat zur lokalen anpassung
    \end{itemize}